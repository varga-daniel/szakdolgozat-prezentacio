%!TEX encoding = UTF-8 Unicode

\documentclass{beamer}

\frenchspacing

\usepackage[utf8]{inputenc}
\usepackage[T1]{fontenc}
\usepackage[magyar]{babel}

% AMS
\usepackage{amssymb,amsmath}

% Graphic packages
\usepackage{graphicx}

% Syntax highlighting
\usepackage{listings}

\usepackage{array}
\newcolumntype{D}{>{\arraybackslash}m{0.3\textwidth}}
\newcolumntype{L}{>{\centering\arraybackslash}m{0.3\textwidth}}
\newcolumntype{Z}{>{\centering\arraybackslash}m{0.6\textwidth}}

\usetheme{Szeged}
\usecolortheme{beaver}

\title[IDE készítése a Rust-hoz]{Integrált fejlesztőkörnyezet készítése a Rust programozási nyelvhez}
\author{Varga Dániel}
\institute{Miskolci Egyetem}
\date{2019. június 12.}

\begin{document}
    \frame{\titlepage}

    \section{Rust}

    \begin{frame}[fragile]
        \frametitle{A Rust nyelv}

        \begin{itemize}
            \item Multi-paradigmájú rendszerprogramozási nyelv
            \item Biztonságos kódra helyez hangsúlyt
            \item Szintaktikailag hasonlít a C++-ra \begin{itemize}
                \item Tervezésben különbözik
                \item Memória-biztonság C++-hoz hasonló sebességek mellett
            \end{itemize}
        \end{itemize}

    \end{frame}

    \begin{frame}[fragile]
        \frametitle{Történet}

        \begin{itemize}
            \item \textbf{2006:} Graydon Hoare, Mozilla alkalmazott személyes projektje
            \item \textbf{2009:} Mozilla támogatja a projektet \begin{itemize}
                \item Létrejön egy csapat, cél: \begin{itemize}
                    \item A nyelv tovább fejlesztése
                    \item A nyelv felhasználása Mozilla Servo projektjében
                \end{itemize}
            \end{itemize}
            \item \textbf{2010:} Mozilla nyilvánosságra hozza a nyelvet \begin{itemize}
                \item A csapat fókuszt vált
                \item Az eredeti OCaml-ben íródott fordító helyett írnak egy Rust fordítót Rust-ban (\texttt{rustc})
                \item LLVM backend-ként
            \end{itemize}
            \item \textbf{2011:} A \texttt{rustc} lefordítja saját magát
            \item \textbf{2012:} Első számozott pre-alfa kiadás
            \item \underline{\textbf{2015. május 15:}} Rust 1.0, nyelv és fordító kiadása \begin{itemize}
                \item Új verziók hat hetente
                \item Három csatornán: alfa, béta, és kiadás
            \end{itemize}
        \end{itemize}
    \end{frame}

    \begin{frame}[fragile]
        \frametitle{Összehasonlítás C++-szal}

        \begin{table}
            \centering
            \begin{tabular}{|D|L|L|}
                \cline{2-3}
                 \multicolumn{1}{c|}{} & Rust & C++ \\ \hline
                 Költség nélküli absztrakció & \multicolumn{2}{|Z|}{Igen: Trait / template dispatchelhető statikusan} \\ \hline
                 Mozgatás szemantika: forrásobjektumok állapota 
                 & Beépített statikus analizáló megtiltja az objektumok használatát mozgatás után 
                 & Javasolt, hogy egy mozgatás konstruktor érvényes állapotban hagyja a forrásobjektumot (de attól azt még nem érdemes használni) \\ \hline
            \end{tabular}
        \end{table}
    \end{frame}
\end{document}
